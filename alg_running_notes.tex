\documentclass{article}


\title{Algorithms Notes}
\author{Karl Pichotta}
\date{Spring 2013}

\usepackage{amsmath,amssymb}


\newcommand{\inv}{^{-1}}
\newcommand{\Z}{\mathbb{Z}}
\newcommand{\I}{\mathcal{I}}

\newcommand{\beq}{\begin{equation}}
\newcommand{\eeq}{\end{equation}}


\begin{document}

\maketitle

\begin{abstract}
These are  notes I've been taking for a graduate algorithms
course taught by Greg Plaxton at UT Austin.
They're taken primarily in-class, and probably only useful for my own
 personal reference.
\end{abstract}

\section{Preamble: Useful identities}




\begin{itemize}

	\item For nonzero $a,b,c$, we have
	\beq
	a^{\log_b c} = 
	c^{\log_b a}
	\eeq

	\item
	For large $n$, we have
	\begin{align}
		\left(
			1 + \frac{1}{n}
		\right)
		&\approx 
		e
		\\
		\left(
			1 - \frac{1}{n}
		\right)
		&\approx 
		e\inv
	\end{align}
	
	\item For nonzero integer $n$, we have
	\beq
	n^2 = n + 2{n\choose 2}.
	\eeq
	Both the algebraic and intuitive justifications for this identity
	should be pretty clear.

\end{itemize}

\section{1/22/2013: Chernoff Bounds, the Tails of the Binomial}

If we have a  binomaial RV $X\sim B(n,p)$, then we have the following
 Chernoff bound, for $0\leq \delta \leq 1$:
$$
P(X\leq (1-\delta) np) \leq
\exp\left\{
	\frac{-\delta^2 np} {2}
\right\}
$$

Supposing we throw $100 n\log n$ balls, by focusing our analysis on a single bin
and then applying Union Bound, we get that the probability that 
some $X_i \leq \log n$ is $\leq n^{-44}$, using the above bound.

\textbf{Fact:}
Thinking about Hashing, suppose we throw $n$ balls into $n$ bins, and we want a bound on the max load of any given bin.
We might hope it's $O(1)$, but it isn't.
Instead, it is
$$
\Theta\left(
	\frac{\log n} {\log \log n}
\right)
$$
with high probability.

\textbf{Upper bound:}
Define $Z = \max_i X_i$, with $X_i$ the number of balls in bin $i$.
Consider bin $1$.
$E[X_1] = 1$, but this tells us nothing about the tail of the distribution,
which is our interest here.

What then, is
$$
Pr\left\{
	X_1 \geq c\frac{\log n}{ \log\log n}
\right\}
$$
Framing it in terms of Chernoff bounds, we will want to use:
$$
\delta = c\frac{\log n}{ \log\log n} - 1 
%\approx
%c\frac{\log n}{ \log\log n} 
$$
We need to use the Large Deviations bound (bound (2) on handout).
This is the following:
Suppose $X\sim Binom(n,p)$, then
$$
Pr\left\{
	X\geq (1 + \delta) np
\right\}
\leq
\left(
	\frac{e^\delta}{(1+\delta)^{1+\delta}}
\right)
^{np}
$$
For concreteness, we take $c=100$.
Our $np=1$, so the RHS of the above chernoff bound is
$$
... \leq
\left(
\frac{e^{\delta + 1}}
{	\left(100 \frac{\log n}{\log \log n} \right) ^ {100 \log n / (\log \log n)}}
\right) 
\leq
\left(
	\frac{\log\log n}{\log n}
\right) ^ {100 \log n / (\log \log n)}
\approx
n^{-100}
$$
WE inflate the numerator by adding an extra $e$.

Sidenote:
Now, this funny $\log\log $n terms come from the fact that solving
$$
x^x = n
$$
gives you something very like $\log n / \log\log n$.
This comes from canceling out lower order approximations to log.

Returning, by union bound, we argue that the probability that some bin gets $\geq 100 \log n / \log\log n$ balls is $\leq n^{-99}$.

\textbf{Lower bound:}
So that was an upper bound. The lower bound is trickier.
Let $k = \varepsilon \log n / \log\log n$, with $\varepsilon$ a small positive argument.
We won't be able to just reason about bin 1 and then use union bound---there is a constant probability $(1 - 1/n)^n \approx e^{-1}$ that bin 1 gets 0 bins.
Let $E_i$ denote the event that bin $i$ receives at least $k$ balls.
We want to show that
$$
Pr(\cup E_i) \geq 1 - \frac{1}{n^c}
$$
This is, reminder, a lower bound on the max.
That is, we're showing that
$$
Z = \Omega(\log n / \log\log n).
$$
We're showing that $Pr(E_1)$ is small, but it'll be useful.

So consider $Pr(E_1)$.
We have
$$
Pr(E_1) \geq Pr(X_1 = k)
$$
(since RHS entails LHS)
This is equal to 
\begin{align}
Pr(X_1 = k)
&= {n \choose k} 
\left(1/n \right)^k
(1 - 1/m)^{n-k}
\\
&\geq
{n\choose k} (1/n)^k
(1/e)
\\
&\geq \left(
	\frac{n}{k}
\right)^k
(1/n)^k
(1/e)
\\
&=
\frac{1}{e k^k}
\\
&\approx \frac{1}{n^{\varepsilon'}}
\end{align}
We have the third to last step by the following useful lower bound:
$$
{n\choose k} \geq (n/k)^k
$$
which you see by expanding the formula and reasoning about the various values of quantities above and below the line.

Now, by linearity of expectation, we expect
$n^{1-{\varepsilon'}}$ bins to get at least $k$ balls.

What we want, however, is a high probability bound.
That is, we want a statement of the form ``at least one of  bins, whp, will get at least $k$ bounds''.
We can't use a Chernoff bound, because the $E_i$'s are not independent, so therefore the distribution isn't Binomial.

From a highl-level perspective: we throw $n$ balls into $n$ bins, then ask if $E_1$ occurred.
If it did, then we'll be happy, because all we want is one of the $E_i$s to occur (NB $PR(E_1) \geq 1 / (n^{\varepsilon'}$).
If $E_1$ doesn't occur, then note
$$
PR(E_2 | E_1^c) \geq Pr(E_2) \geq 1 / {n^{\varepsilon'}}
$$
and so on with all of the $Pr(E_i)$'s.
We can therefore argue that the probbability that no$ E_i$ occurs is
$$
\leq (1 - 1/n^{\varepsilon'})^n
$$
We want to factor this last equation into
\begin{align}
...&=
((1 - 1/n^{\varepsilon'}) ^ {n^{\varepsilon'}} ) ^ {n^{1 - \varepsilon'}}
\\
&\approx
(1/e) n^{1 - \varepsilon'}
\end{align}
Instead of having $1/e$ to some logarithmic quantity, we have it to some polynomial quantity.
So this is much, much less than inverse polynomial bound, and we've more than satisfied our sharp threshold.

An exercise close to the one we did at the begnning---using Chernoff bounds,
we can argue that the number of flips of a fair coin to get $\log n$ heads
with high probability is
$$
\Theta(\log n)
$$
(we use Chernoff bound (1), as in the first example).
This comes up in at least one way to analyze randomized Quicksort.

In last class, we saw that the expected number of comparisons is $\Theta(n\log n)$.
We now argue that, with high probability, the number of comparisons is $O(n\log n)$.
Note this is quite different: what we say now is that the probability that you 
exceed the expected runtime by a factor of, say, 50, is very small.
Note these RVs aren't binomial, but we'll be able to bound their behavior
with binomial RVs.

First, we'd LIKE to show the number of comparisons involving a specific key is $O(\log n)$ with 
high probability.
This is not true---there is a $1/n$ chance of an element being the first
pivot, and it gets compared to $n$ elements.

So what we'll do instead is to use a charging scheme: we charge
a comparison to a non-pivot.
Whenever we make a comparison, ``charge'' the comparison to the nonpivot (each comparison involves a pivot and a nonpivot).
It IS true that the charge to any key is $O(\log n)$ with high probability.
Once we show that, it immediately follows that the total number of comparisons is $O(n\log n)$ WHP.
HOw can we use the prvious result to convince ourselves of this?



\section{1/24/2013: Hashing}

Let's think about Randomized quicksort.
We know that randomized quicksort has expected $O(n\log n)$ behavior.
(that is, there are no ``bad inputs'' for it in expectation).
Note also that Randomized quicksort has $O(n\log n)$ runtime with high
probability (that is, we'll give a nice bound on the probability of 
the runtime being asymptotically higher).


We can think about quicksort from the perspective of a particular fixed
key, call it $x$.
Recall the charging scheme: we ``charge'' each comparison to the nonpivot 
(which we can do, since each comparison is between a pivot and another value).

\textbf{Claim:}
The key $x$ gets $O(\log n)$ charge with high probability (the failure
probability will be $1/n^c$ for arbitrarily large $c$.
(If we prove this, then by Union Bound, we have the total charge to all
$n$ keys is $O(n\log n)$ WHP.)

\textbf{Proof of Claim:}
We have a ``herd'' of keys, initially of size $n$, and select one to be the pivot.
The pivot is remarkably lucky---it gets 0 charge.
If $x$ is chosen as the pivot, that's nice.
If, on the other hand, $x$ is not chosen as pivot, then it will get charge 1.
Then the keys will be partitioned into two ``herds'', one of which will
never again be compared with $x$.

Consider the process where we start with a positive integer $n$.
We flip a coin. 
If we get heads, we set the number to an integer uniformly drawn from
$[0, 3n/4]$ (floored).
If we get tails, then we set the value uniformly to between $0$ and $n - 1$.
So at each stage, the value gets smaller (by at least one).
The process stops when the number gets to 0.

We use this process as follows. 
Picking a pivot at random,
there is a $0.5$ probability that the pivot sits in between $n/4$ and $3n/4$ in the sorted list (it sits in the middle half).
There's also a $.5$ chance it's not in the middle.
The number in the process is just the size of the ``herd'' $x$ is in.

If we get a good pivot, then the herd value is at most $3n/4$ (imagine
getting the rightmost pivot in the middle half, say).
This is like flipping a head.
In the worst case (flipping a tail), then the size of the herd decreases by 1.

The process can't involve more than some constant times $\log n$ heads:
each head decreases the herd size by at least $3/4$.
(Prove this?)
%The maximum number of heads needed to get to 1 is $i$ in
%$$
%\left(
%\frac{3}{4}
%\right)
%^i n = 1
%$$
So $x$ sits there and hopes for good pivot choices from its perspective.
(This is basically the end of the proof.)

We don't have a binomial variable, but we can relate it to this
simpler process defined above, and use a Chernoff bound on that.


\subsection{Hashing with chained overflows}

We have a hashtable, visualized as an array of buckets, numbered $0,\ldots,k-1$.
We have a has function
$$
h(x) \in \{0,\ldots,k-1\}.
$$
In order to handle collisions, we can use a linked list to chain
them together in the bucket.
Searching for an item in the table later involves computing the hash and
then traversing the linked list.

Suppose you put $k$ things in.
You're generally hoping that each bucket tends to have $O(1)$ elements.
In the worst case, you traverse a linked list (and have $O(k)$ lookup).
In the best case, we get $O(1)$ lookup.

What's the runtime of $h$?
Supposing we have $n$ keys, each of which is $10\lg n$ bits.
(Note we're using the RAM  model: the word size of the machine we're
programming is logarithmic in the input size in bits of the problem.
So if we have a million-bit instance, we assume that we can manipulate
words of $\log$ 1m in constant time.)
So that means that our word size is $\Theta(\log n)$ bits.
That is, these keys will fit into a constant number of words.
We'll be relying on this later to assume various operations are $O(1)$.

So for now, we think of $h$ as taking any key and mapping uniformly
random from $0,k-1$.
This is of course not quite right: $h$ must be deterministic.
However, we think of the hash for our purposes as choosing uniformly
at random.

This relates directly to the bin/bucket problems we discussed previously.

\textbf{Claim:}
The average time for a search is $O(1)$, assuming $n$ buckets (with $n$ keys).

No proof (prove this?).
 Basically, the vast majority of buckets will get very few elements.

On the other hand, what's the expected max search time?
That is, has $n$ keys into $n$ buckets with our idealized hash.
The worst-case search time is the longest linked-list we get.
We looked at this last time:
The max load is
$$
\Theta\left(
	\frac{\log n} {\log\log n}
\right)
\,\, w.h.p.
$$
So it's only marginally better than using a red-black tree.

\subsection{Perfect Hashing}

At a high-level, we use \textit{Perfect Hashing} to obviate this
non-constant load problem.
That is the following:

We assume we're dealing with a \textit{static} set of $n$ keys.
That is, we construct a hash-table structure that is specific for the
particular $n$ keys that we have at hand.
As before, assume each key is $10\log_2 n$ bits.

Desiderata:

\begin{itemize}

\item
We want to construct a hash table with $O(1)$ worst case search time.
NOTE this isn't just a matter of expectation: we want guarantees about
the worst case.

\item
We also want to use $O(n)$ space (that is, our solution can't be ``use $O(2^n)$ keys'').

\item
Further, we want ``fast'' construction
\end{itemize}

A naive approach is the following: Repeatedly pick a new hash function
until you find one inducing max load of $O(1)$.
We can do this because we have static keys (we know them in advance).
Eventually we'll find one with $O(1)$ worst case.
However, this won't give us fast construction:
we'll have to run this an exponential number of times in order to 
find this (I don't quite get the proof, but it involves one of the
results we showed last time involving $\log n$).

We'll use a twist on this: we'll follow a similar approach, but our
criterion for selecting a hash function will be looser.
we'll use a two-tiered approach: 
What we'll do is count ``collisions'', and as long as we don't have
more than $n$ collisions, we'll call it good.
So we have $n$ buckets, but instead of a linked list in each bucket,
we have a hash table at each of the $n$ buckets.
That is, we have one primary hash table, and $n$ secondary hash tables.

We want the total size of all these tables to be $O(n)$.
We do our primary hashing; based on how many elements hash into a location,
each bucket has a hash table of that size.
We'll have the size of a secondary hash table being quadratic in the number
of elements there.
So if we have 20 elements in a bucket, we'll have a hash table of size approximately $20^2$.
Why? 
We want to totally avoid collisions at the second level of hashing.
The thing about quadratic size is that this is the threshold at which we'll
suffer 0 collisions in the secondary hash tables (we can repick secondary
hashes).

One concern is that since we're wasting space in the secondary tables,
when we add the size of the secondary tables up, it'll be too large.
So we need to show that the sum of the table sizes will be linear
(intuitively, we'll show the vast majority of buckets get only a constant number
of keys).

So OK getting to details. What is a ``collision''?
Mapping keys to buckets, we'll let $Y_i$ denote the RV corresponding
to the number of keys mapping to bucket $i$ at the top level (for $1\leq i\leq n$).
Let $X$ be the RV denoting the number of collisions.
Then
$$
X = \sum_{i=1}^n {Y_i\choose 2}.
$$
That is, if $10$ keys are mapped to a bucket, that gives 10 choose 2 pairwise
collisions.
Note that
$$
X = \Theta\left(
	\sum_{i=1}^n Y_i^2
\right)
$$
which is, magically, the total size of the secondary hash tables.
So $X$ is constant-factor-related to the space required for the secondary
hash tables.
(Note that it's good enough, then, to pick a hash function that induces
at most $100n$ or $1000n$ collisions.)

Now, to get at $E[X]$, we express $X$ as a sum of indicator variables.
Define $Z_{ij}$ to be the indicator variable taking 1 if the $i$, $j$th keys
collide, and 0 otherwise.
Note there are $n$ choose $2$ such vars.
So
\begin{align}
E[X] 
&=
\sum_{i,j} E[Z_{ij}]  \\
&=
\sum_{i,j} 1/n \\
&=
{n \choose 2} / n
\\
&= 
\frac{n-1}{2}
\end{align}
where we appeal to the fact that we have an idealized hash function.
This is less than $1/2$ of our target of getting $\leq n$ collisions.
(It is important that we choose a constant $> 1/2$ when defining
our criterion for accepting a hash functions.)

Call a hash function ``good'' if it induces $\leq n$ collisions; ``bad'' otherwise.
On average, the number of collisions is $n/2$.
We can therefore bound the percentage of all hash functions that
are ``bad''.
If, for example, 90\% of the has functions are bad, then the expected
number of collisions would be at least $0.9n$.
So we have that at most half of the hash functions are bad.
Note that this is a huge overestimate: the only way
this can happen (that is, we get the expected value we had before)
 is if all the bad hash functions have exactly $n$ collisions and the
 good ones have $0$.

So in our first phrase, picking a primary hash function,
This is, in the worst case, like flipping a fair coin until you get
a heads.
So the expected number of trials is $O(1)$ (it is Geometric(0.5)).
So, with high probability, the number of trials is $O(\log n)$ (proof?).

Once we have the top-level hash function, then, the $Y_i$ values are determined.
In bucket $i$, we use a hash table of size $Y_i^2$.
We repeatedly pick hash functions for bucket $i$ until we find one
that gives no collisions at all.

That is, the secondary hash table at buekct $i$ has $Y_i^2$ size, $Y_i$ keys.
The expected number of collisions is calculated using a similar approach
before.
We define ${Y_i\choose 2}$ indicator variables, and the probability that
one of those is $1$ is $1 / Y_i^2$.
So defining $C_i$ th enum of collisions at $i$, we have
$$
E[C_i] = 
{Y_i\choose 2} (1 / Y_i^2)
=
\frac{Y_i - 1}{2Y_i}
\leq \frac{1}{2}.
$$
The analysis is identical for every bucket.
So cool! We're done, this scheme works.

\subsection{Realistic Hash Functions}

The tricky thing here is that we've been assuming that we have these
idealized has functions that distribute uniformly, and we can generate
them easily (and they're independent).
What is it like in real life?


When we were talking about idealized hash functions, the associated
family corresponds to the family of
$$
n^{n^{10}}
$$
hash functions (this is the number of functions from $10\lg n$ strings to
$n$ things, I think?).
Note if we represent these naively, then we use base-2 log num bits to
specify an element.
This is probematic, because this description of a function is of 
length $n^{10}\lg 10$ bits.

The key thing to note here (that'll get us around this)is that pairwise 
independence is sufficient
for the family of hash functions we use.
What do we mean?
We mean the following:
A family of hash functions $\mathcal H$ is pairwise independent if,
for any distinct keys $x,y$, and any (possibly equal) bucket
indices $i,j$, if the hash function $h$ is drawn uniformly at random
from $\mathcal H$,
then
$$
Pr(h(x) = i\,\,\&\,\, h(y) = j)
=
\frac{1}{B}
$$
with $B$ the number of buckets.

This is sufficient because, in our analysis, we were only interested in
pairwise indicator variables. 
We were never interested in any more complicated conditions.



We're interested in designing a family of functions mapping from
$(10 \log_2 n)$-bit strings to $(\log_2 n)$-bit strings.

We saw taht for analysis, we didn't need full independence, but just
pairwise independence.
This is because our analysis cared only about the number of pairwise collisions,
which we can write as a sum of indicator variables $X_{ij}$, which is
1 iff $i$ and $j$ collide.
Recall that $P\{X_{ij} = 1\} = 1/p$, with $p$ the number of buckets.

In our analysis, in fact, it's fine to have approximate pairwise independence;
that can give us OK bounds.
So we want a function $h$ such that, for two distinct keys $x\neq y$ (and any $i,j$, not necessarily distinct),
then
$$
Pr\{
	h(x) = i \,\wedge\,
	h(y) = j
\}
=
O\left(
	\frac{1}{n^2}
\right).
$$
IN the above, $i$ and $j$ are bins; $x$ and $y$ are keys.

First, we'll pick a prime $p$ a bit bigger than $n^{10}$ (why?).
The prime number theorem tells us that about, picking things around the
neighborhood, we only have to try a logarithmic number of keys (in $n^{10}$)
in expectation before finding a prime.
(There is also a theorem indicating that there is definitely a prime
between $k$ and $2k$; in particular, this is $n^{10}$ and $2n^{10}$).

So let's hash from $\Z_p$ to $\Z_n$.
Consider
$$
h(x) = \left[ ax + b \mod p\right]
\mod n
$$
with $a$ and $b$ chosen from $\Z_p$.
(Note that we have $a$ and $b$ because we want it to be the case that,
under our analysis, there will be no ``bad input''---if we have an adversary
picking keys, they won't be able to pick bad keys if they know what the
family looks like; if there were no $a$ and $b$, then they would be
able to do so. In other words, our family needs to have more than one
hash function in it if we want to argue a small number of expected
collisions.)
This isn't quite uniform between $0$ and $n-1$, but we're quite close.
The inner hash
$$
ax + b \mod p
$$
is, I think, actually uniform over $\Z_p$ (proof?).
The outer hash, then, will be approximately uniform over $\Z_m$ (proof?).
This satisfies the condition for approximate pairwise independence
that we defined earlier.

So at the top level, we pick a $p$, then we repeatedly choose $a$ and $b$; for each
$a$ and $b$, we check how the particular hash function works.
We repeatedly do that until we get a good $a$ and $b$; we store them,
as they fully parametrize a hash function.
We do this for each primary hash function and each secondary hash function.





\section{1/29/2013: Dynamic Programming}

\subsection{Examples}

\subsubsection{contiguous subarrays of booleans}
Suppose we're given an $n\times n$ boolean array $A$.
We want to find the side-length of a largest all-true contiguous square
subarray of $A$.
So if there's a $3\times 3$ subarray of Trues, then the desired answer
is at least 3.

Now, certainly this problem is solvable in polynomial time.
For each side length $k$, there are only $n^2$ different places where the
$k\times k$ subarrays top-left corner could be; we could
just exhaustively check this for all $k$ and compute the answer.

We can get a better polynomial bound using Dynamic Programming.
What DP does is, instead of solving a single problem instance, find a bunch
of problem instances to solve, and do so in such a way that the larger
problem instances can use the computations performed for the smaller
instances.

So let $a_{ij}$ be the size of the largest all-true contiguous
square subarray with lower right corner
at location $(i,j)$.
This is a family of $n^2$ subproblems.
Our answer to the original problem is just
$$
\max_{i,j} a_{ij}.
$$

Now, what is a good order to solve these $n^2$ subproblems?
What we'd like is that, whenever we get to a particular problem
in this ordering, we can very easily solve it in terms of the instances
we've already solved.
In other words, we want to be able to write a suitable reccurrence for
the $a_{ij}$'s.
Thinking about it, it's not too tough to see the following works:
\begin{equation}
a_{ij} = 
\left\{
\begin{array}{ll}
0 & \textrm{if } (i,j) \textrm{ entry is F}
\\
1 + \min(a_{i-1,j}, a_{i,j-1}, a_{i-1,j-1}) & \textrm{otherwise}
\end{array}
\right.
\end{equation}
So filling in left-to-right row by row (or top-down column by column) should
work.
The important thing is that, when we get to a position, 
the spot above it, to the left of it, above and to the left of it, are
filled in.
(A small note is that we need to interpret out-of-bounds indices as 0 above.)

So the above algorithm runs in $O(n^2)$ time, which is a nice,
much-faster polynomial-time algorithm than the brute-force poly-time algorithm.

\subsubsection{Rod-cutting problem}

Commonly, Dynamic programming yields polynomial-time algorithms for
problems where the brute-force technique is exponential.
For example, suppose we have a rod of length $n$ inches ($n\in \Z^+$).
We can cut it into any integer-length pieces (such that they sum to $n$).
For every $i$, we have a price $p_i$ that we can sell a rod of length $i$
for.
The problem, then, is to cut up the rod into pieces in such a way that
we maximize the total prices for the pieces.

When we cut the number up, we get a multiset of positive integers; in
the language of number theory, this is a \emph{partition} of $n$.
If the number of partitions of $n$ were sufficiently slow, we could just
generate them and calculate their value; however, the number of partitions
grows superpolynomially.
Consider the numbers
$$
1,2,\ldots, \lceil 2\sqrt n \rceil - 1,
\lceil 2\sqrt n \rceil.
$$
Pair up the first two numbers and the last two numbers:
$$
1,2,\ldots, \lceil 2\sqrt n \rceil - 1,
\lceil 2\sqrt n \rceil.
$$
For the first $\lceil 2\sqrt n\rceil + 1$ elements, you can either
partition them into the two outermost pieces or the next
innermost pieces.

We repeat for $3,4$ and the next two inner numbers on the right.
For each of these, we eat up $\sqrt n$ of the rod.
We'll have $\sqrt n / 2$ such decisions for what to do with the
pieces of length $\sqrt n$, and we'll have about
$$
2^{\Omega(\sqrt n)}
$$
ways to partition the rods up.
(In fact, it ends up being $2^{O(\sqrt n)}$ too.)
So the number of partitions of $n$ is pretty clearly superexponential.

We can solve this, however, with a simple one-dimensional dynamic program.
For each $i$, let $v_i$ be the maximum value of a rod of
length $i$.
We will use the following recurrence for $v_i$ in terms of the smaller $v_j$'s.
For a partition of $i$, there will be some length $\ell$ of the last
partition.
So
\begin{align*}
v_0 &= 0 \\
v_i &=
\max_{1\leq \ell \leq i}
v_{i-\ell} + p_\ell
\end{align*}
We get $O(n^2)$ time of this: we have $n$ subproblems, but the $i$th
subproblem takes $O(i)$ time to solve (summing over these subproblems
uses the familiar $\sum_i i^2 = O(n^2)$ identity).


\subsubsection{Longest Common Subsequence}

We have two sequences $X,Y$ of symbols over some finite alphabet.
Writing $X$ as
$$
X = x_1 x_2 \dots x_n,
$$
we define a subsequence of $X$ as a subset of entries $x_i$, concatenated
in order.
We want to find the largest integer $k$ such that there are
subsequences in $X$ and $Y$ of length $k$ such that both subsequences are
the same.

The brute-force approach here is exponential: it's pretty easy to see (there are $2^n$ subsets of $n$ sets).

We use the familiar two-dimensional DP solution.
Let $X_i$ denote the length-$i$ prefix of $X$:
$$
X_i = x_1\dots x_i,
$$
and similarly with $Y_j$.
Now, let
$a_{ij}$ be the length of the LCS of $X_i$ and $Y_j$.

Supposing $|X| = m$ and $|Y| = n$, we have $m\times n$ subproblems.
The final answer to the question is $a_{mn}$.
So we have the base cases
\begin{align*}
a_{0j} &= 0 \\
a_{i,0} &= 0
\end{align*}
for all $i,j$.
For the recursive case, we have
\begin{equation}
a_{ij} =
\left\{
	\begin{array}{ll}
		a_{i-1,j-1} + 1 & \mathrm{if } x_i = y_j \\
		\max\{a_{i,j-1}, a_{i-1,j}\} & \mathrm{if } x_i \neq y_j.
	\end{array}
\right.
\end{equation}




\subsubsection{The Partition Problem (a pseudopolynomial algorithm)}

This problem is NP-hard.
Suppose we have $n$ positive integers $x_1, \ldots, x_n$.
We want to know whether the $x_i$'s can be partitioned into two multisets
of equal sum.

We give a DP algorithm.
First, let $\sum x_i = 2s$.
(Note if the sum is odd, we return that there is no such partition).
To determine $s$, it' a natural thing to consider computing
\beq
a_{i,j} = \left\{
\begin{array}{ll}
	T & \textrm{if there's a subseq. of $x_1,\ldots,x_i$ summing to exactly $j$} \\
	F & o.w.
\end{array}
\right.
\eeq
Note $\forall i$, $a_{i,0} = T$.
Further, $\forall j > 0$ $a_{0,j} = F$.
When $i,j > 0$, we have
\beq
a_{i,j} = \left\{
\begin{array}{ll}
	a_{i-1,j} & 
	\textrm{if } j < x_i
	\\
	a_{i-1,j} \vee a_{i-1,j-x_i}
	& o.w.
\end{array}
\right.
\eeq
NB we only really need the first term for the technicality of the subscript
on the RHS disjunct being negative.

Now, the number of table entries is $O(nS)$.
Is this polynomial time?
Intuitively, no---$S$ can be very large.
We argue that the algorithm is not polynomial; to do so, we need to find
one family of inputs where the runtime is not upper-bounded by a polynomial
function of the input in bits.

What do we mean by polynomial time?
Well, we upper-bound the runtime of an algorithm according to a polynomial
function of its input in bits.
What's the input of this like?
Well, consider an input where each $x_i$ is an $n$-bit integer.
The input size for this problem is $\Theta(n^2)$ bits (we have $n$ $n$-bit numbers).
For us to claim that we have a polynomial runtime, we need that, for such
inputs, the runtime is polynomial in $n$.
$S$, however, is $O(n 2^n)$, since an $n$-bit integer can be as large
as $2^n$ in value (and we have $n$ of them).
So, more importantly, $S$ can be $\Omega(n 2^n)$.
So this algorithm, which kinda looks like it's polynomial, actually
isn't polynomial in the input size.

However, sometimes these sorts of algorithms are useful.
If we know, for example, that each of the numbers is at most $n^3$, then we have
that $S$ is at most $n^4$, and we have a polytime algorithm.
We call algorithms like this \emph{pseudopolynomial}: they are
polynomial in the input size if the input integers are represented
in unary.


\section{1/31/2013: Greedy Algorithms}

\subsection{A Scheduling Problem}

Suppose we have $n$ tasks, each with a positive integer deadline and an
execution requirement.
So an input may look like

\begin{tabular}{| lll |}
\hline
task & Deadline & execution requirement \\
A & 8 & 5 \\
B & 6 & 2 \\
C & 9 & 2 \\\hline
\end{tabular}

We want to know if we can meet all the deadlines in a nonpreemptive schedule?
So, in the above, suppose we run $B$ at time 0 and $C$ at time 2; both
of these take time 2, and have met their deadlines. 
If we run $A$ next, though, it'll miss its deadline.
However, if we run in the order $BAC$, we get a feasible schedule.

We may want to try to get at this by Dynamic Programming, but we don't
need it: a greedy solution ends up sufficing.
We use \emph{earliest deadline}, breaking ties arbitrarily.
This will yield a feasible schedule if one exists.

We want to argue that the answer to the greedy algorithm is right iff there's
a feasible schedule.
Clearly, Left to Right is easy.
The argument is R to L, then: If there's a feasible schedule, then we need
to show the greedy algorithm finds it.

So take a feasible schedule.
Assume the feasible schedule has no ``spaces'': there's no advantage
to idle time, so everything is ``squeezed'' as far as possible to the left.
Suppose that we have a schedule $ABCD$ of four elements, in that order.
Suppose that $C$ has an earlier deadline than $B$; earliest deadline
would have chosen the order $ACBD$, rather than $ABCD$.

We can modify the schedule by leaving everything the same except for 
$B$ and $C$, swapping the latter two.
Note the new schedule is clearly the same length.
We want to show that $ACBD$ is also feasible.
Clearly $C$ meets it deadline still---it's moved earlier.
We need to show that $B$ is meeting its deadline.
Since, by assumption, $C$ had a more stringent deadline than $B$, it must
be the case that if $B$ finishes when $C$ did before, it certainly meets
its deadline, since it meets $C$'s deadline.
In other words, before, with $ABCD$, $C$ was meeting its deadline; we know that 
$B$'s deadline is later than $C$'s, so in $ACBD$, $B$ definitely meets its deadline.

We apply the same logic to the new problem instance to argue that it's feasible 
(eventually we will get to the earliest-first schedule).




\subsection{A variation of the scheduling problem}

Suppose that, in addition to the deadline and execution requirements,
each task has a positive profit $p_i > 0$ associated with it.
Our goal is to find a max-profit feasible subset (that is, we won't in general
use the entire set of tasks).
What we'll do is combine the greedy approach with dynamic programming.

We first order tasks by deadline (in nondecreasing order).
We then consider all prefixes with respect to that particular ordering.
The DP's subproblem is $a_{k,t}$, defined as the max profit we can
obtain with a schedule using only the first $k$ tasks and time $\leq t$.
We can write a recurrence fairly straightforwardly (left as an exercise).

The runtime of the above will be polynomial if the execution requirements
are polynomially-bounded.
That is, if every task has a task that is $O(n^c)$ for some $c$.
Without such an assumption, we could get an exponentially-sized
table in our DP


\subsection{Matroids}

Captures a large class of greedy algorithms.
Consider the MInimal Spanning Tree problem, in particular, Kruskal's
greedy algorithm.
One way to convince ourselves that Kruskal's algorithm is correct
is to reason directly about minimal spanning trees.
Another is to frame the algorithm as a matroid problem, and use a general
property about Matroids.

A Matroid is a pair
$$
M = (S,\mathcal I)
$$
with $S$ analogous to the vertices in a graph: it's just a set.
$\mathcal I$ is called the \emph{independent sets} of $S$: $\mathcal I \subseteq 2^S$ (elements are subsets of $S$).
We need two more properties:
\begin{enumerate}
	\item
	\textbf{Hereditary Property}: if $A\in\mathcal I$ and $B\subseteq A$, 
	then $B\in\mathcal I$.
	
	\item
	\textbf{Exchange Property}:
	If $A,B\in \mathcal I$ and $|A| > |B|$, then
	there is some $x \in A\setminus B$ such that $B\cup\{x\} \in \mathcal I$.
\end{enumerate}
The hereditary property tells us that removing elements from an independent
set yields an independent set.
In particular, this tells us that $\emptyset\in\mathcal I$.

Define a maximal independent set as an independent set such that, if you
add any additional elements to it, it will no longer be independent.
The exchange property tells us that all maximal independent sets
will have the same cardinality.
A maximal independent set is sometimes called a \textbf{basis}

In a \textbf{weighted matroid}, each element $x\in S$ has an associated
weight $w(x)$.
In applications, we often want to compute a maximum (or minimum) weight
maximal independent set.
For example, eventually we'll have a minimal spanning tree representing a
minimum weight maximal independent set (the maximal independent sets will
correspond to the spanning trees).




\subsection{Matroid Greedy Algorithm}

Suppose we want to do maximization.
\begin{itemize}

\item
First, sort the elements of $S$ in nonincreasing order of weight (if we
are minimizing, we'll sort in the other direction).

\item
Initialize
$
A := \emptyset
$

\item
For each $x\in S$, in the order determined in the first step:
	\begin{itemize}
		\item
		If $A\cup \{x\} \in \mathcal I$, then $A := A \cup \{x\}$.
	\end{itemize}
\end{itemize}
And that's it.
Thinking about Kruskal's algorithm, the set $\mathcal I$ will be the set
of acyclic graphs---the conditional above corresponds to the conditional
in that algorithm.


How do we show this algorithm is correct?
First, index the elements of $S$ $1,\ldots,n$ according to the
ordering in the first step.
Call the weight for element $i$ in this ordering $w_i$.
For each $i$, calculate some max-weight maximial independent set $B$.
Mark down if element $i$ is in $B$ or not.
Also, for each $i$, consider $A$, the set computed at that step by the algorithm.
What we need to do is prove, for each step, $A$ is a maximal weight independent step.
Similarly, look at whether elem $i$ is put in $A$.

Look at the first part where the sets $A$ and $B$ differ.
It can't be the case that the elem $i$ is in $B$ but not in $A$ (proof? It has
somethign to do with the hereditary property, not sure what).
So the first place where the two differ is that an element $i$ must
be in $A$ and not in $B$.
WE note the first place $i$ where $A$ and $B$ differ.
Construct $A'\subset A$ from all the before $i$ in the ordering, and also elem $i$.
Now, we know $|B| \geq A'$, since $|A'| \leq |A|$, and $|B| = |A|$, by the
exchange property.

We repeatedly apply the exchange property to add elements from $B$ to $A'$.
The exchange property tells us that we can add an element from $B$ to $A'$
and get an independent set, so long as $|A'| < |B|$.
This process terminates exactly when $|A'| = |B|$.

When we're done growing $A'$, $B$ gave $A'$ all its elements from $i+1$ onwards
except for one of them.
In other words,
$$
A' = (B + i) = j
$$
for some $j$.
That is $A'$ has elem $i$ while $B$ doesn't, and $A'$ deosn't get some
$j>i$.
However, since the elements are ordered by weight, and $w_j \leq w_i$,
we can conclude that
$$
w(A') \geq w(B).
$$

Now we play the whole game over, but with $B$ replaced by $A'$.
This gets us ``one step'' in the right direction, and we iterate
this process until we end up with $A' = A$, at which point we can
argue that $A$ was a max-weight maximal independent set.



\subsection{Some examples of Matroids}

\begin{itemize}
	\item \textbf{Uniform matroid}: Let $|S| = n$. 
	Fix $0\leq k \leq n$.
	$\mathcal I$ is the set of all subsets of $S$ of size $\leq k$.
	This is a matroid.
	The hereditary property is pretty obvious.
	The exchange property is also pretty clear: if we have $A$ and $B$
	such that $|A| < |B|$, then there's clearly an element in $B$ that's
	not in $A$ such that adding it to $A$ will result in an element of 
	cardinality $\leq k$.
	
	
	\item
	\textbf{Partition matroid}: 
	Fix a partition of $S$ into $\{S_1,\ldots, S_k\}$.
	Define $\mathcal I$ to be the set of all subsets $A\subseteq S$
	such that $\forall i |S_i\cap A| \leq 1$.
	
	A basis (maximal independent set) for this matroid
	would be a set that chooses exactly one element from each $S_i$.
	The number of such bases is
	$$
	\prod_{i=1}^k |S_i|.
	$$
	The hereditary property is pretty clear.
	The exchange property is a bit trickier, but also pretty clear.
%	Suppose we have two sets $A$ and $B$ with 
%	$A$ smaller. The argument is a bit more complicated for this and
%	I'm not coyping it down, but not too tough.
	
	\item
	\textbf{Scheduling problem}: we have $n$ tasks
	numbered from $1$ to $n$.
	Task $i$ has a positive integer deadline $d_i$, a positive weight
	$w_i$ (the profit of the task, say), and an execution requirement
	of 1 (unit tasks). We're scheduling these tasks on a shared resource.
	
	What is the maximum-weight schedulable subset of the tasks (in general,
	we might not be able to meet all tasks of the schedule)?
	This is a bit complex: Suppose there are 2 tasks with deadline 1, 
	and two tasks with deadline 2. We can schedule
	at most one of the former, but two of the latter (but only if we
	schedule neither of the first two).
	
	This problem ends up having a useful matroid structure, allowing us to
	use the matroid greedy algorithm.
	
	Define the set $S$ as the set of $n$ tasks.
	$\mathcal I$ will be the schedulable subsets of the tasks.
	If this works, our optimal solution will certainly be a maximal 
	independent set (this isn't necessarily true if we have negative
	weights I don't think).
	
	Now, to show $(S,\mathcal I)$ is a matroid. The hereditary property
	is pretty clear.
	The exchange property is a bit trickier.
	Suppose $A,B\in\mathcal I$, with $|A| > |B|$.
	We proved earlier that $A$ is schedule iff any ED schedule (???) of
	$A$ is feasible (wtf does this mean?).
	Define a ``canonical'' ED schedule as one of the ED schedules: break
	ties in a particular manner.
	That is, we're constructing a schedule of a task $A$; we list tasks
	in increasing order of their index or something like that, guaranteeing
	a unique tie-breaking schedule.
	The schedule, then, will be some ordering
	$$
	A = x_{i_1}, x_{i_2}, \ldots, x_{i_{|A|}}
	$$
	and
	$$
	B = x_{i_1}, x_{i_2}, \ldots x_{i_{|B|}}.
	$$
	Note that there will be no ``empty spaces'' in the schedule.
	
	Let's focus on the last task appearing in $A$ that isn't a part of $B$.
	Define $z$ to be this tasks: the last task in the canonical ED schedule
	of $A$ that does not belong to $B$.
	To show the exchange property, we need to show there's something in $A$
	that can be added to $B$.
	We'll claim that $z$ is such an element; that is,
	$B+z \in\mathcal I$ (that is, $B + z$ is schedulable).
	
	\begin{itemize}
		\item \textbf{Case 1}: $z$ is the last task in the ED schedule of $A$.
		Because $B$ is shorter, we can just tack $z$ onto the end of $B$;
		it will surely meet its deadline, because it will start at least
		as early as it started in $A$.
		This isn't necessarily an ED schedule, but is feasible.
		
		\item
		\textbf{Case 2}:
		The ED schedule of $A$ ends with
		$
		z, y_1,\ldots,y_\ell
		$
		with $\ell \geq 1$ (that is, $z$ isn't the last element of $A$'s ED
		schedule).
		What we do is jimmy around with the schedule of $B$ to arrive at
		a feasible schedule for $B+z$.
		Leave the schedule before $y_1$ alone; put $z$ in the spot where
		$y_1$ is.
		Find $y_2$ in $B$'s schedule, and put $y_1$ in it; similarly,
		put $y_{\ell - 1}$ into the schedule where $y_\ell$ was, and
		tack $y_\ell$ onto the end (recall, these are the occurrences
		of $y_1,\ldots,y_\ell$ in $B$'s ED schedule.
		Now, $y_\ell$ in the new schedule for $B$ is no further to 
		the right than it is in $A$'s schedule.
		Similarly, the furthest to the right that $y_{\ell - 1}$ could
		possibly be is where it is in $A$'s schedule.
		So this is a feasible schedule (KBP: i don't see why 
		$y_1,\ldots, y_\ell$ are in $B$'s schedule. Maybe they don't
		have to be?).
	\end{itemize}
	
	
	
	
	
	
	\item
	\textbf{Scheduling with release times}:
	a more complicated version of the above.
	Assume each task also has a positive integer \textbf{release time}.
	Previously, we assumed all tasks were available at time $0$; we
	restrict that assumption now: tasks become available to executable
	at a certain point in time, and can only be scheduled past that time.
	
	Even with this additional constraint, we can construct a matroid
	structure (we don't here).
	
	\item
	\textbf{Transversal matroids}: even more general.
	Say we have a bipartite graph $(U,V,E)$ (with $U,V$ the two sets of
	vertices such that $\forall (u,v)\in E$, $u$ and $b$ aren't both in
	$U$ or $V$).
	Consider the matroid $(U,\mathcal I)$, with every $S\in\mathcal I$ a
	``matchable'' subset of $U$, defined below.
	A \textbf{matching} is a subsets of the edges $E$ that induces degree
	at most $1$ on every vertex.
	That is, in a matching has no vertex incident on more than 1 edge (that is,
	every vertex has degree $\leq 1$).
	Given a matching, call a vertex \textbf{matched} if it's incident
	on one of the edges in the matching.
	A subset $U$ of vertices is \textbf{matchable} if there exists some
	matching matching every $u\in U$.
	
	So our matroid is $(U,\mathcal I)$, with $U$ the same set in our
	bipartite graph $(U,V,\mathcal I)$ (proof that this is a matroid?).
	Relating to the last scheduling problem: Consider $U$ the set of
	tasks, and $V$ the set of times during which a task can start.
	So supposing task $i$ has deadline of $5$ and a release time of $2$ (and
	takes $2$ time units),
	we'll add edges from $i\in U$ to $2,3,4$ in $V$.
	Given this setup, every matchable subset corresponds to a valid
	scheduling.
	
	The hereditary property is easy: if a set of tasks is matchable,
	of course any subset is matchable.
	
	The exchange property is tricky: say we have $A,B\in\mathcal I$,
	$|A| > |B|$.
	Let $M_A$ be a matching matching exactly $A$ (which exists by assumption), 
	and
	let $M_B$ be a matching for $B$.
	That is, the set of nodes matching in $M_A$ is exactly $A$ (so
	$|M_A| = |A|$), ditto $B$.
	We need to show that there is some $M$ matching $B+x$, for some
	$x\in A\setminus B$.
	Think of these matchings as sets of edges.
	
	Consider the graph $G\equiv (U,V,M_A \Delta M_B)$ (taht is, the symmetric
	difference of $M_A$ and $M_B$: edges in exactly one of $M_A, M_B$).
	Because $M_A$ is such that each node is of degree at most 1, we
	have that $M_A\Delta M_B$ induces degree of at most 2.
	$G$ consists of vertex-disjoint paths and cycles (the cycles are of
	even length; each cycle would alternate between an edge in $M_A$ and
	and edge in $M_B$---you can't have two consecutive edges from $M_A$,
	because this would give us a node of degree 2).
	
	We cannot have only cycles: since we know that $|M_A| > |M_B|$, we must
	have some paths (since every cycle draws an equal number of
	edges from $M_A$ and $M_B$, I think).
	In particular, there must be at least one odd-length path that
	begins and ends with an edge in $M_A$.
	Note that the paths, like the edges, alternate from edges in
	$M_A$ and edges in $M_B$.
	So assume WLOG that this path starts in $U$ and ends in $V$ (it's
	odd-length, and a bipartite graph, so an odd-length path can't
	begin and end in $U$ or $V$).
	Call the node in $U$ that the path starts at $x$.
	
	We can obtain a matching for $B + x$ by making a slight
	hack on $M_B$.
	Take the odd-length path above; get rid of the original edges
	from $M_B$, and add the edges from $M_A$ that were touching
	the same nodes.
	This doesn't lose matchability (every one of the nodes on the
	path is still connected to exactly one edge, this time the edge
	from $M_A$), and also now $x$ is matched.
	
	It's worth noting that the computational problem to solve the matroid
	greedy algorithm is trickier for this formulation: we have to determine
	matchability for arbitrary subsets, which is a little tricky.
	
	\item
	\textbf{Graphic matroids}:
	Fix an undirected graph $G=(V,E)$.
	Define $M = (S,\mathcal I)$, with $S = E$
	and
	$\mathcal I$ the set of all acylic subsets of $E$.
	The hereditary property is pretty obvious: removing edges won't
	add cycles.
	
	We don't have time for the exchange property in class.
	
	
\end{itemize}
 



\section{2/7/2013: Minimal Spanning Trees and Matroids}

Recall that, given an undirected graph $G=(V,E)$, we construct a graphic
matroid from it by the set $(E,\I)$, with $A\in\I$ iff $A$ is acyclic.
Here, the Maximally independent set are exactly the spanning forests;
if $G$ is connected, then it is a spanning tree.

Note that a \textbf{minimal dependent set} of a matroid $(S,\I)$ 
is a dependent set (that is, a subset of $S$ not in $\I$) such that if we
remove any element from it, we get an independent set.
If we have a graphic matroid, then a minimal dependent set will be a cycle
(we can't throw any edge out of the set without getting something
that's acyclic).

\subsection{A theorem and some corollaries about Matroids}

\textbf{Lemma}:
let $M=(S, \I)$ be a weighted matroid.
Let $A,B$ be distinct max-weight bases of
$M$.
Then there exists a sequence $A_1, \ldots, A_k$ of max-weight bases of $M$
such that $A=A_1, B=A_k$, and for every $i$ with $1\leq i < k$, we have
$$
|A_{i+1} \setminus A_i| = 1.
$$
That is, $A_{i+1}$ and $A_i$ differ by a ``swap'':
$$
A_{i+1} = A_i + x - y
$$
with $x$ and $y$ having equal weight (since they're all max-weight).

\textbf{Proof}: 
How do we prove this?
One way is to use something similar to our proof of correctness for the
matroid greedy algorithm.
We order the elements $(x_i, w_i)$ by decreasing order
of weight, and then we line up $A$ and $B$, marking whether
$x_i \in A$ and $x_i \in B$. We look at the first
element such that $x_i\in A$ but $x_i\not \in B$, and, we obtain
$B'$ by growing an $A^*$ somehow (how?!),
 starting with $A^*$, and growing to a basis via repeated application of the exchange
property with $B$.
So we'll end up with $B'$, which will differ from $B$ by a single swap.
There'll be one element that $B$ does have but that $B'$ doesn't have;
other than that, $B'$ (which is, I think, what you get after constructing
as many $A^*$s as you can) and $B$ will be identical.

Since $A$ and $B$ are both max-weight bases, what can we conclude
about the basis $B'$?
We have that $w(B') \geq w(B)$; we can get $B'$ from $B$ by exchanging
the elements that differ, and the one in $B'$ is at least as heavy as that
in $B$ (why?!).
We cannot have $w(B') > w(B)$, since $B$ is a max-weight basis.

So we will have obtained a third max-weight basis that agrees on a longer
prefix with $A$ than $B$ did.
We can do this again, getting a max-weight basis $B''$ differing from $B'$ by
a single edge swap; eventually, we will end up with something
equal to $A$.
We will therefore have exhibited a sequence of the desired form: Something
starting at $B$ and ending up at $A$, where each sequence is a max-weight basis
differing from the previous by a single element.


\textbf{Corollary to above:} all max-weight bases have the same distribution 
of weights (that is, taking just the element weights and sorting it
will yield the same list).
(You might think that, e.g., you can have two minimal spanning trees
such that one has an edge of weight 10 and 5, and the other has instead
two edges of weight 8 and 7; this corollary rules out that possibility.)

\textbf{Another corollary}: Every max-weight basis is a
possible output of the matroid greedy algorithm (that is, will be produced
if ties in weight are resolved appropriately).
You might think that, in Kruskal's algorithm, for example, that you can
have a graph with a couple MSTs such that one of them will never be produced;
this isn't possible, by this corollary. There will be a tie-breaking that will
cause every MST to be produced.
It's pretty easy to see what tie-breaking is appropriate: break ties by
giving priority to elements in our desired max-weight basis.

\textbf{A third corollary}: if the weights are all distinct, then there
is a unique max-weight basis.
Cool!

\subsection{Matric Matroids}

There is another class of matroids called \textbf{``matric matroids''},
defined in terms of a given matrix $A$.
Define such a matroid $M=(S,\I)$, with $S$ the columns of $A$, and
$\I$ the sets of linearly independent columns of $A$.
You can't model all matroids as matric matroids, but a bunch of the classes
of matroids discussed above can be modeled as matric matroids.

It's useful, in general, to consider these special classes of matroids,
rather than considering matroids in full generality, because we can
often get faster algorithms for specific types of matroids.


\section{2/7/2013 (cont): Amortized Analysis}

Amortized analysis is typically used in settings where the cost of
operations is nonuniform: we ask, if we spend $k$ operations, what sort
of bounds can we put on performance.
So we may have some operations that are much more expensive than others, but
nevertheless we may be able to bound the overall cost as being less than the 
pessimistic estimate (wherein we upper-bound
every single operation by the worst-case performance).

There are different methods of amortized analysis:
\begin{enumerate}

	\item 
	\textbf{Aggregate method}:
	Simply sum up the cost of all the individual operations to bound the total
	cost of a sequence of operations.
	This is the most basic approach, but can get kind of messy.
	
	\item
	\textbf{Accounting method}: 
	Suppose we're trying to prove a theorem that any sequence of $k$ 
	operations
	will take $O(k)$ time.
	Then, when an operation is performed, we'll 
	give an operation a constant amount of dollars in order to do it. 
	For example, we'll
	give an operation $\$10$ to fund its execution, if we're trying to bound
	the operations at amortized $\$10$.
	The actual semantics are that an operation uses the money we give it; if
	it uses less than we give it, it puts the rest of the money into different
	piles of money in the data structure.
	When an expensive operation takes a lot, then it uses the leftover piles
	of money from the preceding operations.
	
	If the money we inject into a system is enough to pay for all the 
	operations,
	then we can get bounds on the running time.
	
	Note that it'll frequently make sense to convert this approach
	to the potential approach, described below (where the potential
	corresponds to ``amount of money left in data structure'').
	
	\item
	\textbf{Potential method}: (sometimes, potential function method):
	A more general method, emphasized more in this course.
	We define a potential function mapping the state of the 
	data structure to a (usually nonnegative) number.
	The amortized cost of an operation will be defined as the actual cost plus
	the change in potential.
	
	So if an operation changes the state of the data structure from
	$D$ to $D'$, then the change in potential is $\phi(D') - \phi(D)$, with
	$\phi$ the potential.
	
	We'll use the potential function to make sure the amortized costs all
	look similar, making it easy to add up the amortized cost.
	
	This seems weird---what we really care about is the total cost, rather
	than the total amortized cost (which is this funny function of the
	actual cost that we made up).
	However, summing things up makes it make sense:
	
	Suppose the state of our data structure before/after each of 
	our operations is $D_0, \ldots, D_k$.
	When we sum the total amortized cost, we get the potential telescoping: 
	the sum
	of the potentials is 
	$$
	\phi(D_1) - \phi(D_0) + \phi(D_2) - \phi (D_1) + \ldots
	= \phi(D_k) - \phi(D_0)
	$$
	which gives us that the total amortized cost is the
	total cost plus the final potential ($\phi(D_k)$ minus the initial potential ($\phi(D_0)$).
	
	Usually, the initial potential is $0$, and the final potential is
	typically nonnegative; this means that the total amortized cost,
	according to this method, will actually be an overestimate of the
	total cost (since total amortized cost = total cost + final potential).
	So if we can place an upper bound on the total amortized cost, that will
	also give us an upper bound on the total cost.
	
\end{enumerate}




\subsection{Three Examples}

\subsubsection{Push/multipop stack}

We have a cheap operation, push (push a single item from the stack),
and a potentially expensive one, multipop, which pops $k$ times 
(taking time proportional to $k$).
Multipop($k$) can only be performed on a stack with at least $k$ elements.

So a push has actual cost $1$, and multipop($k$) will have cost $k$.
What is the cost of a sequence of $n$ operations starting with
the empty stack?

\begin{itemize}
	\item 
	\textbf{Naive approach}:
	At all times, we cannot have more then $n$ elements on the stack.
	No single operation costs more than $n$: at most, we can pop $n$
	elems at any time.
	Since we have $n$ operations, the total cost is $\leq n^2$.
	
	This is a very weak upper bound: we'll see that our handful of
	expensive operations must be offset by a bunch of cheap 
	operations.
	
	\item
	\textbf{Aggregate Method}:
	The total cost of all multipops must be $\leq$ the total cost of all 
	pushes.
	If we ever multipop off 10 items, we can only do that if, previously,
	we'd pushed all of them on, at a total cost of $n$.
	Since the total cost of all pushes is $\leq n$ (we perform at most
	$n$ pushes of cost 1), we have that the total cost is at most $2n$.
	
	\item
	\textbf{Accounting method}:
	Give the system $\$2$ for each push, and $\$0$ for each multipop.
	We'll want to show that the pushes have amortized cost 2 and the
	multipops have 0 amortized cost; this would imply that the total
	amortized cost is at most $2n$.
	
	When a push happens, it has to take one dollar; it puts the
	extra dollar onto the pile.
	At any point, a stack of $m$ elements will have $\$m$ sitting around.
	In fact, we can think of each element keeping its dollar, and so
	we can easily prove that each element has one dollar associated with
	it, and so we'll be able to fund every operation.
	
	\item
	\textbf{Potential method}:
	Define $\phi$ of a stack to be the number of elements on the stack
	(notice that this is basically the same as the accounting argument).
	Then it's easy to show, using the argument we just gave, that
	we can bound the cost of operations linearly.
	

\end{itemize}




\subsubsection{Incrementing a binary counter}

We have a counter that has the number $0$ initially, and we have only
one operation: increment counter.
We increment in the most elementary way: start at the LSB, flip and
possibly carry, and continue until you hit a 0.
So the increment operations are occasionally expensive (if you have to
flip a lot of 1's to 0's), but they're usually cheap.
We'd hope that, over a large sequence of $k$ increments, our total cost
would be $O(k)$.

Concretely, we imagine our counter having an infinite number of bits going
off to the left.
The first increment will take cost 1; the next will take 2 (we have to
flip 2 bits); the third will flip 1 bit; the next 3; and so on.


\begin{itemize}

	\item
	\textbf{Aggregate method}: the total cost of $n$ operations will
	give us taht roughly 1/2 the operations will have cost 1 (every other
	number is even); 1/4 of the ops will cost 2; 1/8 will cost 3, and
	so on.
	So the total cost is
	$$
	(n/2) + 2(n/4) + 3(n/8) + \ldots
	\leq
	\sum_{i\geq 0} \frac{i}{2^i}
	 = 2
	$$
	(the last equivalence is somewhat nontrivial)
	
	\item
	\textbf{Accounting method}: $\$2$ / operation.
	We'll need to maintain a different invariant from the stack: we might
	think that we'd have a dollar per bit, but that won't end up
	working.
	We instad have $\$1$ for each 1-bit in the counter: if our number is
	$001011$, then we have $\$3$.
	It's not tough to see this works: incrementing the counter, if we have
	$k$ bit flips, we had to convert $k-1$ 1's to 0's; we grab that dollar
	going around, and then deposit our extra dollar on the 1.
	
	\item
	\textbf{Potential method}:
	$\phi$ will is the number of 1s present in the counter.
	The amortized cost is the actual cost $+ \Delta\phi$.
	So note that the counter always looks like $01^k$ for some $k$ (with
	some stuff before it);
	an increment flips that to $10^k$.
	So the change in potential is $1 - k$: The potential before is however
	many $1$'s there are; the potential after is $1 - k$ minus that (we
	flip $k$ bits and change 1 to a 0).
	So the actual cost is $k+1$; the change in potential is $1-k$;
	thus the actual cost + $\Delta\phi$ is 2.
	We can thus bound the cost of $n$ increments by $2n$.
	
	Note this is basically also an accounting argument.

\end{itemize}



\subsubsection{Dynamic Array}

The canonical example: suppose we're using an STL vector or a Java
ArrayList.
The underlying implementation doesn't know how much space is needed;
we'll reserve space, but from time to time, the application will
exceed the bounds, and a large cost will be incurred.
So sometimes a single push can be very expensive, but we'll want to 
determine that the total cost of a sequence of operations is, say, $O(k)$.

We want space usage to be proportional to the number of elements in the array;
that is, we don't want to just allocate way more space than we'll ever need
and call it a day.
Further, we'll want to ensure that we're using at least a constant
fraction of our currently allocated array.
Further, we'll want contiguous elements in memory---we can't just 
allocate space on the heap for each element individually.

We'll do the usual thing---when we run out of space, we'll call \texttt{grow},
which doubles the size of the allocation and copies the old elements over.
We'd like to give a potential function argument to prove that the
amortized cost of grow is $O(1)$.
That is, we want a function such that amortized cost = actual cost 
$+ \Delta\phi$
is constant, no matter what phase we're in.

So supposing we have $k$ things in our array, and we have to call grow,
we'll have $\Theta(k)$ cost (say, $ck$).
We then need a potential function such that $\Delta \phi$ is $-ck + O(1)$.
We'll make $\phi$ depend on the current size $s$ (the number of elements
in the array) and capacity $n$ (note $s\leq n$).

If we let
$$
\phi = c( 2s-n)
$$
we note that during a grow operation, we get $\Delta\phi = 2c$, 
or somethign? Maybe I got that wrong (go through this argument!)
Note that after a grow operation, $s=k+1$ and $n=2k$.
Note also that $\phi$ is nonnegative except possibly for the very beginning
(which we could handle with a special case)


Suppose now we want to add a \texttt{shrink} operation: when the
array gets less than a quarter full (that is, $S < n/4$), we want to halve its capacity.
Note we cannot just say ``halve the length of the array when it's less than
half full'', because this would give us a sequence
of extremely expensive operations: we could go back and forth inserting and
removing, right around the point where an array is full, doubling
and halving the array, giving us a large fraction of operations that are
very expensive.

How can we modify $\phi$ such that, with shrink, we still get good performance?
We design a function such that, when we grow, our potential ramps up from
$0$ to $\Theta(n)$ (as the previous one did).
However, we'll also require that when you shrink from half-full to quarter-full,
we want the potential to increase to $\Theta(n)$ too (which our old one didn't
do).
COnsider
$$
\phi = \max\{
	c(2s - n), c'(n - 2s)
\}.
$$
The first covers the case we had before; the latter covers the case
where we're less than half full (the $c$ constants are different because
grow and shrink could have different costs).
This function works---when we grow, we get a big change in potential; when
we shrink, we get a big change in potential; in between, when incrementing
or decrementing, the potential doesn't change much---it's just $O(1)$.
The overall amortized cost, therefore, of these ``noncritical'' operations
is constant.
(Note that sometimes the amortized cost will end up 0 or negative; that's 
fine, just a side-product of the funny math going on.)




























\end{document}